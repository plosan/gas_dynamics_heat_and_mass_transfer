
\subsection{Esquema d'integració temporal}

Els tres esquemes d'integració esmentats anteriorment han estat l'esquema explícit, l'implícit i Crank--Nicolson. Donat que els esquemes implícit i Crank--Nicolson son incondicionalment estables, amb un $\Delta t$ gran donen els mateixos resultats que l'esquema explícit amb un $\Delta t$ petit. A més, com l'esquema de Crank--Nicolson és de segon ordre, al seu torn ha de produir resultats més precisos que l'esquema implícit.

Per l'anàlisi de l'esquema d'integració temporal, es pren de nou una discretització uniforme amb $N_1 = 35$, amb passos de temps $\Delta t \in \{ 10.00, \, 50.00, \, 100.00 \} \ \second$. Els mapes de temperatures calculats són per $t = 5000 \ \second$ i $t = 10000 \ \second$. S'han analitzat els esquemes implícit i de Crank--Nicolson. No s'ha treballat amb l'esquema explícit pels motius esmentats. Pel mètode de Gauss--Seidel, es pren $\delta = 10^{-11}$. 

Els mapes de temperatura per $t = 5000 \ \second$ es recullen a la figura \ref{fig:esquema_5000} i per $t = 10000$ es mostren a la figura \ref{fig:esquema_10000}. L'esquema implícit, tot i ser menys precís, dona tants bons resultats com l'esquema de Crank--Nicolson per a tot $\Delta t$. 