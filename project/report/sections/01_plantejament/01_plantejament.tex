
\section{Introducció}

Es considera una barra molt llarga, de longitud $W$, formada per quatre materials ($M_1$ a $M_4$). A la figura \ref{fig:esquema_problema} es representa la secció transversal de la barra i els diferents materials.
\begin{figure}[h]
	\centering
	\begin{tikzpicture}
		% Fluido azul
		\fill[fill=cyan!70!white] (0,0) rectangle (5*8/11,4*8/11);
		\fill[fill=meatColor!70!white] (5*8/11,0) rectangle (11*8/11,7*8/11);
		\fill[fill=green!70!white] (0,4*8/11) rectangle (5*8/11,8*8/11);
		\fill[fill=red!70!white] (5*8/11,7*8/11) rectangle (11*8/11,8*8/11);
		\draw[black, line width=0.3mm] (5*8/11,0) -- ++(0,8*8/11);
		\draw[black, line width=0.3mm] (0,4*8/11) -- ++(5*8/11,0);
		\draw[black, line width=0.3mm] (5*8/11,7*8/11) -- ++(6*8/11,0);
		\draw[black, line width=0.5mm] (0,0) -- ++(11*8/11,0) -- ++(0,8*8/11) -- ++(-11*8/11,0) -- cycle;
		\draw[->, black, line width=0.5mm] (0,0) -- ++(12*8/11,0) node[above]{$x$};
		\draw[->, black, line width=0.5mm] (0,0) -- ++(0,9*8/11) node[right]{$y$};
		\node[black] at (0.5*5*8/11,0.5*4*8/11) {$M_1$};
		\node[black] at (8*8/11,0.5*7*8/11) {$M_2$};
		\node[black] at (0.5*5*8/11,6*8/11) {$M_3$};
		\node[black] at (8*8/11,7.5*8/11) {$M_4$};
		\node[black] at (-0.3,0) {$p_0$};
		\node[black] at (5*8/11+0.3,4*8/11) {$p_1$};
		\node[black] at (5*8/11-0.3,7*8/11) {$p_2$};
		\node[black] at (11*8/11+.3,8*8/11) {$p_3$};
	\end{tikzpicture}
	\caption{Esquema general del problema.}
	\label{fig:esquema_problema}
\end{figure}

\noindent
La geometria del problema es dona a la taula \ref{tab:coordenades_punts}. Les propietats termofísiques dels materials es donen a la taula \ref{tab:propietats_termofisiques}.
\begin{table}[h]
	\begin{minipage}[t]{.5\linewidth}
		\centering
		\begin{tabular}[t]{lcc}
			\toprule[0.5mm]
			Punt & $x \ [\meter]$ & $y \ [\meter]$ \\
			\midrule[0.25mm]
			$p_0$ & $0.00$ & $0.00$ \\			
			$p_1$ & $0.50$ & $0.40$ \\
			$p_2$ & $0.50$ & $0.70$ \\
			$p_3$ & $1.10$ & $0.80$ \\		
			\bottomrule[0.5mm]
		\end{tabular}
		\caption{Coordenades de $p_0$ a $p_4$}
		\label{tab:coordenades_punts}
	\end{minipage}%
	\begin{minipage}[t]{.5\linewidth}
		\centering
		\begin{tabular}[t]{lccc}
			\toprule[0.5mm]
			Material & $\rho \ [\kilo\gram / \meter^3]$ & $c_p \ [\joule / \kilo\gram \kelvin]$ & $\lambda \ [\watt / \meter \kelvin]$ \\
			\midrule[0.25mm]
			$M_1$ & $1500.00$ & $750.00$ & $170.00$ \\
			$M_2$ & $1600.00$ & $770.00$ & $140.00$ \\
			$M_3$ & $1900.00$ & $810.00$ & $200.00$ \\
			$M_4$ & $2500.00$ & $930.00$ & $140.00$ \\	
			\bottomrule[0.5mm]
		\end{tabular}
		\caption{Propietats termofísiques}
		\label{tab:propietats_termofisiques}
	\end{minipage} 
\end{table}

\noindent
Cada costat de la secció transversal de la barra interactua de manera diferents amb l'entorn, tal com es descriu a la taula \ref{tab:condicions_contorn}:
\begin{table}[h]
	\centering
	\begin{tabular}{ll}
		\toprule[0.5mm]
		Costat 		& Condició de contorn \\
		\midrule[0.25mm]
		Inferior 	& Isoterma a $T = 23.00 \ \celsius$ \\
		Superior 	& Flux de calor uniforme $\dot{Q}_\text{sup} = 60.00 \ \watt / \meter$ en la direcció $y$-negativa \\
		Esquerre 	& En contacte amb un fluid a $T_g = 33.00 \ \celsius$ i un coeficient $\alpha_g = 9.00 \ \watt / \meter^2 \, \kelvin$ \\
		Dret 		& Temperatura uniforme $T = 8.00 + 0.005 t \ \celsius$, on $t$ és el temps en $\second$ \\
		\bottomrule[0.5mm]
	\end{tabular}
	\caption{Condicions de contorn}
	\label{tab:condicions_contorn}
\end{table}

\noindent
Aquest problema és proposat per \cite{cttc}.

%\clearpage
%
%\begin{figure}[h]
%	\centering
%	\begin{tikzpicture}
%		% Fluido azul
%		\path[fill=red!40!white] (0,-1) -- (5,{-5/3-1}) -- (5,{5/3+1}) -- (0,1) -- cycle;
%		\path[fill=cyan!70!white] (0,1) -- (5,{5/3+1}) -- (5,{5/3+1.1}) -- (0,1.1) -- cycle;
%		\path[fill=cyan!70!white] (0,-1) -- (5,{-5/3-1}) -- (5,{-5/3-1.1}) -- (0,-1.1) -- cycle;
%		\draw[scale=1, domain=0:5, smooth, variable=\x, black, line width=0.5mm] plot (\x, {\x/3+1});
%		\draw[scale=1, domain=0:5, smooth, variable=\x, black, line width=0.5mm] plot (\x, {\x/3+1.1});
%		\draw[scale=1, domain=0:5, smooth, variable=\x, black, line width=0.5mm] plot (\x, {-\x/3-1});
%		\draw[scale=1, domain=0:5, smooth, variable=\x, black, line width=0.5mm] plot (\x, {-\x/3-1.1});
%		\draw[blue, dashdotted, black, line width=0.3mm] (-0.5,0) -- (5.5,0);
%		\draw[->, red, line width=0.5mm] (-1.75,0) -- ++(1,0);
%		\draw[->, red, line width=0.5mm] (5.75,0) -- ++(1,0);
%		\draw[<-, blue, line width=0.5mm] (5.75,2.95) -- ++({cos(18.4349)}, {sin(18.4349)});
%		\draw[<-, blue, line width=0.5mm] (-1.75,0.4667) -- ++({cos(18.4349)}, {sin(18.4349)});
%		\draw[<-, blue, line width=0.5mm] (5.75,-2.95) -- ++({cos(18.4349)}, {-sin(18.4349)});
%		\draw[<-, blue, line width=0.5mm] (-1.75,-0.4667) -- ++({cos(18.4349)}, {-sin(18.4349)});
%	\end{tikzpicture}
%\end{figure}
%
%\begin{figure}[h]
%	\centering
%	\begin{tikzpicture}
%		% Fluido azul
%		\path[fill=cyan!70!white] (-0.025,0.25) rectangle (15.025,0.75);
%		\path[fill=cyan!70!white] (-0.025,3.25) rectangle (15.025,3.75);
%		% Fluido rojo
%		\path[fill=red!40!white] (-0.025,1) rectangle (15.025,3);
%		% Estructura
%		\draw[black, line width=0.5mm] (0,0.00) -- ++(15,0) -- ++(0,0.25) -- ++(-15,0) -- cycle;
%		\draw[black, line width=0.5mm] (0,0.75) -- ++(15,0) -- ++(0,0.25) -- ++(-15,0) -- cycle;
%		\draw[black, line width=0.5mm] (0,3.00) -- ++(15,0) -- ++(0,0.25) -- ++(-15,0) -- cycle;
%		\draw[black, line width=0.5mm] (0,3.75) -- ++(15,0) -- ++(0,0.25) -- ++(-15,0) -- cycle;
%		% Patrón rayado
%		\foreach \x in {0,...,59}
%			\draw[black, line width=0.2mm] ({\x*0.25},0) -- ++({0.35*cos(45)}, {0.35*sin(45)});
%		\foreach \x in {0,...,59}
%			\draw[black, line width=0.2mm] ({\x*0.25},0.75) -- ++({0.35*cos(45)}, {0.35*sin(45)});
%		\foreach \x in {0,...,59}
%			\draw[black, line width=0.2mm] ({\x*0.25},3.00) -- ++({0.35*cos(45)}, {0.35*sin(45)});
%		\foreach \x in {0,...,59}
%			\draw[black, line width=0.2mm] ({\x*0.25},3.75) -- ++({0.35*cos(45)}, {0.35*sin(45)});
%		% Ejes
%		\draw[blue, line width=0.3mm, dashdotted] (-0.1,2) -- ++(15.2,0);
%		% Flechas
%		\draw[<-, color=blue, line width=0.5mm] (-0.75,0.5) -- (-0.25,0.5);
%		\draw[<-, color=blue, line width=0.5mm] (15.25,0.5) -- (15.75,0.5);
%		\draw[<-, color=blue, line width=0.5mm] (7.25,0.5) -- (7.75,0.5);
%		\draw[<-, color=blue, line width=0.5mm] (-0.75,3.5) -- (-0.25,3.5);
%		\draw[<-, color=blue, line width=0.5mm] (15.25,3.5) -- (15.75,3.5);
%		\draw[<-, color=blue, line width=0.5mm] (7.25,3.5) -- (7.75,3.5);
%		\draw[->, color=red, line width=0.5mm] (-0.75,2) -- (-0.25,2);
%		\draw[->, color=red, line width=0.5mm] (15.25,2) -- (15.75,2);
%		\draw[->, color=red, line width=0.5mm] (7.25,2) -- (7.75,2);
%	\end{tikzpicture}
%\end{figure}

